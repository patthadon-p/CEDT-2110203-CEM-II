\documentclass[a4paper, 10pt]{article}
\usepackage{../../CEDT-Homework-style}

\usepackage{amsmath}
\allowdisplaybreaks

\setlength{\headheight}{14.49998pt}

\begin{document}
\subject[2110203 - Computer Engineering Mathematics II]
\hwtitle{Optimize 2}{}{Week 7}{6733172621 Patthadon Phengpinij}{ChatGPT (for\,\LaTeX\,styling and grammar checking)}


% ================================================================================ %
\section{Simplex Method \& Two-Phase Method}
% ================================================================================ %


% ================================================================================ %
%                                    Problem 01                                    %
% ================================================================================ %
\begin{problem}
Use the simplex method to solve the following LP.
\[
\begin{array}{@{\hspace{2mm}} r @{\hspace{2mm}} r @{\hspace{2mm}} c @{\hspace{2mm}} l}
    \text{max } & z & = & 3x_1 + 2x_2 \\
    \text{subject to } & x_1 + x_2 & \leq & 3 \\
    & x_1 + 2x_2 & \leq & 5 \\
    & 2x_1 + x_2 & \leq & 5 \\
    & x_1, x_2 & \geq & 0
\end{array}
\]
\end{problem}

\begin{solution}
Re-write the LP into standard form by introducing slack variables \(s_1, s_2, s_3\):
\[
\begin{array}{@{\hspace{2mm}} r @{\hspace{2mm}} r @{\hspace{2mm}} c @{\hspace{2mm}} l}
    \text{max } & z & = & 3x_1 + 2x_2 \\
    \text{subject to } & x_1 + x_2 + s_1 & = & 3 \\
    & x_1 + 2x_2 + s_2 & = & 5 \\
    & 2x_1 + x_2 + s_3 & = & 5 \\
    & x_1, x_2, s_1, s_2, s_3 & \geq & 0
\end{array}
\]

Initial simplex tableau:
\[
\begin{array}{c|ccccc|c}
    \text{Basis} & x_1 & x_2 & s_1 & s_2 & s_3 & \text{RHS} \\
    \hline
    s_1 & 1 & 1 & \textcolor[HTML]{0000FF}{1} & 0 & 0 & 3 \\
    s_2 & 1 & 2 & 0 & \textcolor[HTML]{0000FF}{1} & 0 & 5 \\
    s_3 & 2 & 1 & 0 & 0 & \textcolor[HTML]{0000FF}{1} & 5 \\
    \hline
    z & -3 & -2 & 0 & 0 & 0 & 0 \\
\end{array}
\]

Starting at basic feasible solution \( (x_1, x_2, s_1, s_2, s_3) = (0, 0, 3, 5, 5) \).
\begin{enumerate}
    \item Entering variable: \( x_1 \) (most negative in \( z \)-row).
    \item Leaving variable: \( s_1 \) (minimum ratio test: \( 3/1 = 3 \)).
    \item Pivot on \( (1,1) \) to update tableau.
\end{enumerate}
\[
\begin{array}{c|ccccc|c}
    \text{Basis} & x_1 & x_2 & s_1 & s_2 & s_3 & \text{RHS} \\
    \hline
    x_1 & \textcolor[HTML]{0000FF}{1} & 1 & 1 & 0 & 0 & 3 \\
    s_2 & 0 & 1 & -1 & \textcolor[HTML]{0000FF}{1} & 0 & 2 \\
    s_3 & 0 & -1 & -2 & 0 & \textcolor[HTML]{0000FF}{1} & -1 \\
    \hline
    z & 0 & 1 & 3 & 0 & 0 & 9 \\
\end{array}
\]

Since there are no negative coefficients in the \( z \)-row, the optimal solution is reached:
\[ \boxed{(x_1, x_2) = (3, 0) \text{ with }z = 9}. \]
\end{solution}
% ================================================================================ %

\newpage

% ================================================================================ %
%                                    Problem 02                                    %
% ================================================================================ %
\begin{problem}
Determine whether each of the following LPs is degenerate or nondegenerate.
\end{problem}

% === Problem 2.1. === %
\begin{subproblems_alpha}
    \item
    \[
    \begin{array}{@{\hspace{2mm}} r @{\hspace{2mm}} r @{\hspace{2mm}} c @{\hspace{2mm}} l}
        \text{max } & z & = & 2x_1 + x_2 \\
        \text{subject to } & x_1 + x_2 & \leq & 4 \\
        & x_1 + 2x_2 & \leq & 4 \\
        & x_1, x_2 & \geq & 0
    \end{array}
    \]
\end{subproblems_alpha}

\begin{solution}
Re-write the LP into standard form by introducing slack variables \(s_1, s_2\):
\[
\begin{array}{@{\hspace{2mm}} r @{\hspace{2mm}} r @{\hspace{2mm}} c @{\hspace{2mm}} l}
    \text{max } & z & = & 2x_1 + x_2 \\
    \text{subject to } & x_1 + x_2 + s_1 & = & 4 \\
    & x_1 + 2x_2 + s_2 & = & 4 \\
    & x_1, x_2, s_1, s_2 & \geq & 0
\end{array}
\]

Initial simplex tableau:
\[
\begin{array}{c|cccc|c}
    \text{Basis} & x_1 & x_2 & s_1 & s_2 & \text{RHS} \\
    \hline
    s_1 & 1 & 1 & \textcolor[HTML]{0000FF}{1} & 0 & 4 \\
    s_2 & 1 & 2 & 0 & \textcolor[HTML]{0000FF}{1} & 4 \\
    \hline
    z & -2 & -1 & 0 & 0 & 0 \\
\end{array}
\]

Starting at basic feasible solution \( (x_1, x_2, s_1, s_2) = (0, 0, 4, 4) \).
\begin{enumerate}
    \item Entering variable: \( x_1 \) (most negative in \( z \)-row).
    \item Leaving variable: \( s_1 \) (minimum ratio test: \( 4/1 = 4 \)).
    \item Pivot on \( (1,1) \) to update tableau.
\end{enumerate}
\[
\begin{array}{c|cccc|c}
    \text{Basis} & x_1 & x_2 & s_1 & s_2 & \text{RHS} \\
    \hline
    x_1 & \textcolor[HTML]{0000FF}{1} & 1 & 1 & 0 & 4 \\
    s_2 & 0 & 1 & -1 & \textcolor[HTML]{0000FF}{1} & \textcolor[HTML]{FF0000}{0} \\
    \hline
    z & 0 & 1 & 2 & 0 & 8 \\
\end{array}
\]

Since the RHS of \( s_2 \) is zero, the basic feasible solution is \( \boxed{\textbf{degenerate}} \).
\end{solution}
% ==================== %


% === Problem 2.2. === %
\begin{subproblems_alpha}[start=2]
    \item
    \[
    \begin{array}{@{\hspace{2mm}} r @{\hspace{2mm}} r @{\hspace{2mm}} c @{\hspace{2mm}} l}
        \text{max } & z & = & 2x_1 + x_2 \\
        \text{subject to } & x_1 + x_2 & \leq & 4 \\
        & x_1 + 2x_2 & \leq & 6 \\
        & x_1, x_2 & \geq & 0
    \end{array}
    \]
\end{subproblems_alpha}

\begin{solution}
Re-write the LP into standard form by introducing slack variables \(s_1, s_2\):
\[
\begin{array}{@{\hspace{2mm}} r @{\hspace{2mm}} r @{\hspace{2mm}} c @{\hspace{2mm}} l}
    \text{max } & z & = & 2x_1 + x_2 \\
    \text{subject to } & x_1 + x_2 + s_1 & = & 4 \\
    & x_1 + 2x_2 + s_2 & = & 6 \\
    & x_1, x_2, s_1, s_2 & \geq & 0
\end{array}
\]

Initial simplex tableau:
\[
\begin{array}{c|cccc|c}
    \text{Basis} & x_1 & x_2 & s_1 & s_2 & \text{RHS} \\
    \hline
    s_1 & 1 & 1 & \textcolor[HTML]{0000FF}{1} & 0 & 4 \\
    s_2 & 1 & 2 & 0 & \textcolor[HTML]{0000FF}{1} & 6 \\
    \hline
    z & -2 & -1 & 0 & 0 & 0 \\
\end{array}
\]

\newpage

Starting at basic feasible solution \( (x_1, x_2, s_1, s_2) = (0, 0, 4, 6) \).
\begin{enumerate}
    \item Entering variable: \( x_1 \) (most negative in \( z \)-row).
    \item Leaving variable: \( s_1 \) (minimum ratio test: \( 4/1 = 4 \)).
    \item Pivot on \( (1,1) \) to update tableau.
\end{enumerate}
\[
\begin{array}{c|cccc|c}
    \text{Basis} & x_1 & x_2 & s_1 & s_2 & \text{RHS} \\
    \hline
    x_1 & \textcolor[HTML]{0000FF}{1} & 1 & 1 & 0 & 4 \\
    s_2 & 0 & 1 & -1 & \textcolor[HTML]{0000FF}{1} & 2 \\
    \hline
    z & 0 & 1 & 2 & 0 & 8 \\
\end{array}
\]

Since there is no zero in the RHS, the basic feasible solution is \( \boxed{\textbf{nondegenerate}} \).
\end{solution}
% ==================== %


% === Problem 2.3. === %
\begin{subproblems_alpha}[start=3]
    \item
    \[
    \begin{array}{@{\hspace{2mm}} r @{\hspace{2mm}} r @{\hspace{2mm}} c @{\hspace{2mm}} l}
        \text{max } & z & = & 2x_1 + x_2 \\
        \text{subject to } & x_1 + x_2 & \leq & 4 \\
        & x_1 + 2x_2 & \leq & 8 \\
        & x_1, x_2 & \geq & 0
    \end{array}
    \]
\end{subproblems_alpha}

\begin{solution}
Re-write the LP into standard form by introducing slack variables \(s_1, s_2\):
\[
\begin{array}{@{\hspace{2mm}} r @{\hspace{2mm}} r @{\hspace{2mm}} c @{\hspace{2mm}} l}
    \text{max } & z & = & 2x_1 + x_2 \\
    \text{subject to } & x_1 + x_2 + s_1 & = & 4 \\
    & x_1 + 2x_2 + s_2 & = & 8 \\
    & x_1, x_2, s_1, s_2 & \geq & 0
\end{array}
\]

Initial simplex tableau:
\[
\begin{array}{c|cccc|c}
    \text{Basis} & x_1 & x_2 & s_1 & s_2 & \text{RHS} \\
    \hline
    s_1 & 1 & 1 & \textcolor[HTML]{0000FF}{1} & 0 & 4 \\
    s_2 & 1 & 2 & 0 & \textcolor[HTML]{0000FF}{1} & 8 \\
    \hline
    z & -2 & -1 & 0 & 0 & 0 \\
\end{array}
\]

Starting at basic feasible solution \( (x_1, x_2, s_1, s_2) = (0, 0, 4, 8) \).
\begin{enumerate}
    \item Entering variable: \( x_1 \) (most negative in \( z \)-row).
    \item Leaving variable: \( s_1 \) (minimum ratio test: \( 4/1 = 4 \)).
    \item Pivot on \( (1,1) \) to update tableau.
\end{enumerate}
\[
\begin{array}{c|cccc|c}
    \text{Basis} & x_1 & x_2 & s_1 & s_2 & \text{RHS} \\
    \hline
    x_1 & \textcolor[HTML]{0000FF}{1} & 1 & 1 & 0 & 4 \\
    s_2 & 0 & 1 & -1 & \textcolor[HTML]{0000FF}{1} & 4 \\
    \hline
    z & 0 & 1 & 2 & 0 & 8 \\
\end{array}
\]

Since there is no zero in the RHS, the basic feasible solution is \( \boxed{\textbf{nondegenerate}} \).
\end{solution}
% ==================== %
% ================================================================================ %

\newpage

% ================================================================================ %
%                                    Problem 03                                    %
% ================================================================================ %
\begin{problem}
Use the two-phase method to solve the following LP.
\[
\begin{array}{@{\hspace{2mm}} r @{\hspace{2mm}} r @{\hspace{2mm}} c @{\hspace{2mm}} l}
    \text{max } & z & = & 2x_1 + 3x_2 \\
    \text{subject to } & 3x_1 + x_2 & \leq & 14 \\
    & x_1 - x_2 & \geq & 1 \\
    & -x_1 + 3x_2 & \leq & 2 \\
    & x_1, x_2 & \geq & 0
\end{array}
\]
\end{problem}

\begin{solution}
Re-write the LP into standard form by introducing slack, excess, and artificial variables \(s_1, s_2, e_1, a_1\):
\[
\begin{array}{@{\hspace{2mm}} r @{\hspace{2mm}} r @{\hspace{2mm}} c @{\hspace{2mm}} l}
    \text{max } & z & = & 2x_1 + 3x_2 \\
    \text{subject to } & 3x_1 + x_2 + s_1 & = & 14 \\
    & x_1 - x_2 - e_1 + a_1 & = & 1 \\
    & -x_1 + 3x_2 + s_2 & = & 2 \\
    & x_1, x_2, s_1, s_2, e_1, a_1 & \geq & 0
\end{array}
\]

Using Two-Phase Method, we first solve Phase 1 by minimizing the sum of artificial variables:
\[ \text{min } w = a_1 \]

Initial simplex tableau for Phase 1:
\[
\begin{array}{c|cccccc|c}
    \text{Basis} & x_1 & x_2 & s_1 & e_1 & a_1 & s_2 & \text{RHS} \\
    \hline
    s_1 & 3 & 1 & \textcolor[HTML]{0000FF}{1} & 0 & 0 & 0 & 14 \\
    a_1 & 1 & -1 & 0 & -1 & \textcolor[HTML]{0000FF}{1} & 0 & 1 \\
    s_2 & -1 & 3 & 0 & 0 & 0 & \textcolor[HTML]{0000FF}{1} & 2 \\
    \hline
    w & 0 & 0 & 0 & 0 & 1 & 0 & 0 \\
\end{array}
\]

Becase we have to eliminate \( a_1 \) from the \( w \)-row, we perform the row operation:
\[ \text{R}_w \leftarrow \text{R}_w - \text{R}_{a_1} \]


The updated tableau:
\[
\begin{array}{c|cccccc|c}
    \text{Basis} & x_1 & x_2 & s_1 & e_1 & a_1 & s_2 & \text{RHS} \\
    \hline
    s_1 & 3 & 1 & \textcolor[HTML]{0000FF}{1} & 0 & 0 & 0 & 14 \\
    a_1 & 1 & -1 & 0 & -1 & \textcolor[HTML]{0000FF}{1} & 0 & 1 \\
    s_2 & -1 & 3 & 0 & 0 & 0 & \textcolor[HTML]{0000FF}{1} & 2 \\
    \hline
    w & -1 & 1 & 0 & 1 & 0 & 0 & -1 \\
\end{array}
\]

Starting at basic feasible solution \( (x_1, x_2, s_1, e_1, s_2, a_1) = (0, 0, 14, 0, 2, 1) \).
\begin{enumerate}
    \item Entering variable: \( x_1 \) (most negative in \( w \)-row).
    \item Leaving variable: \( a_1 \) (minimum ratio test: \( 1/1 = 1 \)).
    \item Pivot on \( (2,1) \) to update tableau.
\end{enumerate}
\[
\begin{array}{c|cccccc|c}
    \text{Basis} & x_1 & x_2 & s_1 & e_1 & a_1 & s_2 & \text{RHS} \\
    \hline
    s_1 & 0 & 4 & \textcolor[HTML]{0000FF}{1} & 3 & -3 & 0 & 11 \\
    x_1 & \textcolor[HTML]{0000FF}{1} & -1 & 0 & -1 & 1 & 0 & 1 \\
    s_2 & 0 & 2 & 0 & -1 & 1 & \textcolor[HTML]{0000FF}{1} & 3 \\
    \hline
    w & 0 & 0 & 0 & 0 & 1 & 0 & 0 \\
\end{array}
\]

Since there are no negative coefficients in the \( w \)-row and \( w = 0 \), we proceed to Phase 2 by removing the artificial variable \( a_1 \) and solving the original objective function.

\newpage

Initial simplex tableau for Phase 2:
\[
\begin{array}{c|ccccc|c}
    \text{Basis} & x_1 & x_2 & s_1 & e_1 & s_2 & \text{RHS} \\
    \hline
    s_1 & 0 & 4 & \textcolor[HTML]{0000FF}{1} & 3 & 0 & 11 \\
    x_1 & \textcolor[HTML]{0000FF}{1} & -1 & 0 & -1 & 0 & 1 \\
    s_2 & 0 & 2 & 0 & -1 & \textcolor[HTML]{0000FF}{1} & 3 \\
    \hline
    z & -2 & -3 & 0 & 0 & 0 & 0 \\
\end{array}
\]

Becase we have to adjust the \( z \)-row according to the current basis, we perform the row operations:
\[\text{R}_z \leftarrow \text{R}_z + 2\text{R}_{x_1} \]

The updated tableau:
\[
\begin{array}{c|ccccc|c}
    \text{Basis} & x_1 & x_2 & s_1 & e_1 & s_2 & \text{RHS} \\
    \hline
    s_1 & 0 & 4 & \textcolor[HTML]{0000FF}{1} & 3 & 0 & 11 \\
    x_1 & \textcolor[HTML]{0000FF}{1} & -1 & 0 & -1 & 0 & 1 \\
    s_2 & 0 & 2 & 0 & -1 & \textcolor[HTML]{0000FF}{1} & 3 \\
    \hline
    z & 0 & -5 & 0 & -2 & 0 & 2 \\
\end{array}
\]

Starting at basic feasible solution \( (x_1, x_2, s_1, e_1, s_2) = (1, 0, 11, 0, 3) \).
\begin{enumerate}
    \item Entering variable: \( x_2 \) (most negative in \( z \)-row).
    \item Leaving variable: \( s_2 \) (minimum ratio test: \( 3/2 = 1.5 \)).
    \item Pivot on \( (3,2) \) to update tableau.
\end{enumerate}
\[
\renewcommand{\arraystretch}{1.3}
\begin{array}{c|ccccc|c}
    \text{Basis} & x_1 & x_2 & s_1 & e_1 & s_2 & \text{RHS} \\
    \hline
    s_1 & 0 & 0 & \textcolor[HTML]{0000FF}{1} & 5 & -2 & 5 \\
    x_1 & \textcolor[HTML]{0000FF}{1} & 0 & 0 & -\frac{3}{2} & \frac{1}{2} & \frac{5}{2} \\
    s_2 & 0 & \textcolor[HTML]{0000FF}{1} & 0 & -\frac{1}{2} & \frac{1}{2} & \frac{3}{2} \\
    \hline
    z & 0 & 0 & 0 & -\frac{9}{2} & \frac{5}{2} & \frac{19}{2} \\
\end{array}
\]

Then,
\begin{enumerate}
    \item Entering variable: \( e_1 \) (most negative in \( z \)-row).
    \item Leaving variable: \( s_1 \) (minimum ratio test: \( 5/5 = 1 \)).
    \item Pivot on \( (1,4) \) to update tableau.
\end{enumerate}
\[
\renewcommand{\arraystretch}{1.3}
\begin{array}{c|ccccc|c}
    \text{Basis} & x_1 & x_2 & s_1 & e_1 & s_2 & \text{RHS} \\
    \hline
    s_1 & 0 & 0 & \frac{1}{5} & \textcolor[HTML]{0000FF}{1} & -\frac{2}{5} & 1 \\
    x_1 & \textcolor[HTML]{0000FF}{1} & 0 & 0 & 0 & -\frac{1}{10} & 4 \\
    s_2 & 0 & \textcolor[HTML]{0000FF}{1} & 0 & 0 & \frac{3}{10} & 2 \\
    \hline
    z & 0 & 0 & 0 & 0 & \frac{7}{10} & 14 \\
\end{array}
\]

Since there are no negative coefficients in the \( z \)-row, the optimal solution is reached:
\[ \boxed{(x_1, x_2) = (4, 2) \text{ with }z = 14}. \]
\end{solution}
% ================================================================================ %

\newpage

% ================================================================================ %
%                                    Problem 04                                    %
% ================================================================================ %
\begin{tosubmit}
\problem[4] Formulate the following problem into LP and solve it.
You can use any method you want.

\vspace{3mm}

Hamtaro likes to eat sunflower seed. He wants to eat it as much as possible. There are two
types of sunflower seed: the regular one and the low-fat one. A gram of regular sunflower
seed contains 0.5g of fat, 0.2g of protein, and 0.1g of fiber. A gram of low-fat sunflower seed
contains 0.3g of fat, 0.3g of protein, and 0.15g of fiber.

\vspace{3mm}

Hamtaro should get at most 11 grams of fat and at most 8 grams of protein per day. To be
healthy, he should get at least 3 grams of fiber per day. Find the maximum possible total
amount of sunflower seed he can eat in a day.

\vspace{3mm}

\par\noindent\submitsolution
Let \( x_1 \) be the grams of regular sunflower seed and \( x_2 \) be the grams of low-fat sunflower seed that Hamtaro eats in a day.
The objective is to maximize the total amount of sunflower seed he can eat:
\[ \text{max } z = x_1 + x_2 \]

The constraints based on fat, protein, and fiber intake are as follows:
\[
\begin{array}{@{\hspace{2mm}} r @{\hspace{2mm}} r @{\hspace{2mm}} c @{\hspace{2mm}} ll}
    \text{subject to } & 0.5x_1 + 0.3x_2 & \leq & 11 & \text{(fat constraint)} \\
    & 0.2x_1 + 0.3x_2 & \leq & 8 & \text{(protein constraint)} \\
    & 0.1x_1 + 0.15x_2 & \geq & 3 & \text{(fiber constraint)} \\
    & x_1, x_2 & \geq & 0 & \text{(non-negativity constraint)}
\end{array}
\]

Since, the third constraint is a `greater than or equal to' type, we should use the two-phase method to solve this LP.

\vspace{3mm}

\par\hspace{5mm} Re-write the standard form by introducing slack variables \( s_1, s_2 \), excess variable \( e_1 \), and artificial variable \( a_1 \):
\[
\begin{array}{@{\hspace{2mm}} r @{\hspace{2mm}} r @{\hspace{2mm}} c @{\hspace{2mm}} l}
    \text{max } & z & = & x_1 + x_2 \\
    \text{subject to } & 0.5x_1 + 0.3x_2 + s_1 & = & 11 \\
    & 0.2x_1 + 0.3x_2 + s_2 & = & 8 \\
    & 0.1x_1 + 0.15x_2 - e_1 + a_1 & = & 3 \\
    & x_1, x_2, s_1, s_2, e_1, a_1 & \geq & 0
\end{array}
\]

\par\hspace{5mm} Using Two-Phase Method, we first solve Phase 1 by minimizing the sum of artificial variables:
\[ \text{min } w = a_1 \]

Initial simplex tableau for Phase 1:
\[
\begin{array}{c|cccccc|c}
    \text{Basis} & x_1 & x_2 & s_1 & s_2 & e_1 & a_1 & \text{RHS} \\
    \hline
    s_1 & 0.5 & 0.3 & \textcolor[HTML]{0000FF}{1} & 0 & 0 & 0 & 11 \\
    s_2 & 0.2 & 0.3 & 0 & \textcolor[HTML]{0000FF}{1} & 0 & 0 & 8 \\
    a_1 & 0.1 & 0.15 & 0 & 0 & -1 & \textcolor[HTML]{0000FF}{1} & 3 \\
    \hline
    w & 0 & 0 & 0 & 0 & 0 & 1 & 0 \\
\end{array}
\]

Becase we have to eliminate \( a_1 \) from the \( w \)-row, we perform the row operation:
\[ \text{R}_w \leftarrow \text{R}_w - \text{R}_{a_1} \]

The updated tableau:
\[
\begin{array}{c|cccccc|c}
    \text{Basis} & x_1 & x_2 & s_1 & s_2 & e_1 & a_1 & \text{RHS} \\
    \hline
    s_1 & 0.5 & 0.3 & \textcolor[HTML]{0000FF}{1} & 0 & 0 & 0 & 11 \\
    s_2 & 0.2 & 0.3 & 0 & \textcolor[HTML]{0000FF}{1} & 0 & 0 & 8 \\
    a_1 & 0.1 & 0.15 & 0 & 0 & -1 & \textcolor[HTML]{0000FF}{1} & 3 \\
    \hline
    w & -0.1 & -0.15 & 0 & 0 & 1 & 0 & -3 \\
\end{array}
\]

\newpage

Starting at basic feasible solution \( (x_1, x_2, s_1, s_2, e_1, a_1) = (0, 0, 11, 8, 0, 3) \).
\begin{enumerate}
    \item Entering variable: \( x_2 \) (most negative in \( w \)-row).
    \item Leaving variable: \( a_1 \) (minimum ratio test: \( 3/0.15 = 20 \)).
    \item Pivot on \( (3,2) \) to update tableau.
\end{enumerate}
\[
\renewcommand{\arraystretch}{1.3}
\begin{array}{c|cccccc|c}
    \text{Basis} & x_1 & x_2 & s_1 & s_2 & e_1 & a_1 & \text{RHS} \\
    \hline
    s_1 & 0.3 & 0 & \textcolor[HTML]{0000FF}{1} & 0 & 2 & -2 & 5 \\
    s_2 & 0 & 0 & 0 & \textcolor[HTML]{0000FF}{1} & 2 & -2 & 2 \\
    a_1 & \frac{2}{3} & \textcolor[HTML]{0000FF}{1} & 0 & 0 & -\frac{20}{3} & \textcolor[HTML]{0000FF}{1} & 20 \\
    \hline
    w & 0 & 0 & 0 & 0 & 0 & \frac{20}{3} & 0 \\
\end{array}
\]

Since there are no negative coefficients in the \( w \)-row and \( w = 0 \), we proceed to Phase 2 by removing the artificial variable \( a_1 \) and solving the original objective function.

\vspace{3mm}

Initial simplex tableau for Phase 2:
\[
\renewcommand{\arraystretch}{1.3}
\begin{array}{c|ccccc|c}
    \text{Basis} & x_1 & x_2 & s_1 & s_2 & e_1 & \text{RHS} \\
    \hline
    s_1 & 0.3 & 0 & \textcolor[HTML]{0000FF}{1} & 0 & 2 & 5 \\
    s_2 & 0 & 0 & 0 & \textcolor[HTML]{0000FF}{1} & 2 & 2 \\
    x_2 & \frac{2}{3} & \textcolor[HTML]{0000FF}{1} & 0 & 0 & -\frac{20}{3} & 20 \\
    \hline
    z & -1 & -1 & 0 & 0 & 0 & 0 \\
\end{array}
\]

Becase we have to adjust the \( z \)-row according to the current basis, we perform the row operation:
\[ \text{R}_z \leftarrow \text{R}_z + \text{R}_{x_2} \]

The updated tableau:
\[
\renewcommand{\arraystretch}{1.3}
\begin{array}{c|ccccc|c}
    \text{Basis} & x_1 & x_2 & s_1 & s_2 & e_1 & \text{RHS} \\
    \hline
    s_1 & 0.3 & 0 & \textcolor[HTML]{0000FF}{1} & 0 & 2 & 5 \\
    s_2 & 0 & 0 & 0 & \textcolor[HTML]{0000FF}{1} & 2 & 2 \\
    x_2 & \frac{2}{3} & \textcolor[HTML]{0000FF}{1} & 0 & 0 & -\frac{20}{3} & 20 \\
    \hline
    z & -\frac{1}{3} & 0 & 0 & 0 & -\frac{20}{3} & 20 \\
\end{array}
\]

Starting at basic feasible solution \( (x_1, x_2, s_1, s_2, e_1) = (0, 20, 5, 2, 0) \).
\begin{enumerate}
    \item Entering variable: \( e_1 \) (most negative in \( z \)-row).
    \item Leaving variable: \( s_2 \) (minimum ratio test: \( 2/2 = 1 \)).
    \item Pivot on \( (2,5) \) to update tableau.
\end{enumerate}
\[
\renewcommand{\arraystretch}{1.3}
\begin{array}{c|ccccc|c}
    \text{Basis} & x_1 & x_2 & s_1 & s_2 & e_1 & \text{RHS} \\
    \hline
    s_1 & 0.3 & 0 & \textcolor[HTML]{0000FF}{1} & 0 & 0 & 3 \\
    s_2 & 0 & 0 & 0 & \frac{1}{2} & \textcolor[HTML]{0000FF}{1} & 1 \\
    x_2 & \frac{2}{3} & \textcolor[HTML]{0000FF}{1} & 0 & \frac{10}{3} & 0 & \frac{80}{3} \\
    \hline
    z & -\frac{1}{3} & 0 & 0 & \frac{10}{3} & 0 & \frac{80}{3} \\
\end{array}
\]

\newpage

Then,
\begin{enumerate}
    \item Entering variable: \( x_1 \) (most negative in \( z \)-row).
    \item Leaving variable: \( s_1 \) (minimum ratio test: \( 3/0.3 = 10 \)).
    \item Pivot on \( (1,1) \) to update tableau.
\end{enumerate}
\[
\renewcommand{\arraystretch}{1.3}
\begin{array}{c|ccccc|c}
    \text{Basis} & x_1 & x_2 & s_1 & s_2 & e_1 & \text{RHS} \\
    \hline
    s_1 & \textcolor[HTML]{0000FF}{1} & 0 & \frac{10}{3} & 0 & 0 & 10 \\
    s_2 & 0 & 0 & 0 & \frac{1}{2} & \textcolor[HTML]{0000FF}{1} & 1 \\
    x_2 & 0 & \textcolor[HTML]{0000FF}{1} & -\frac{20}{9} & \frac{10}{3} & 0 & 20 \\
    \hline
    z & 0 & 0 & \frac{10}{9} & \frac{10}{3} & 0 & 30 \\
\end{array}
\]

Since there are no negative coefficients in the \( z \)-row, the optimal solution is reached:
\[ \boxed{(x_1, x_2) = (10, 20) \text{ with } z = 30}. \]
\end{tosubmit}
% ================================================================================ %


\end{document}