\documentclass[a4paper, 10pt]{article}
\usepackage{../../CEDT-Homework-style}

\usepackage{amsmath}
\allowdisplaybreaks

\setlength{\headheight}{14.49998pt}

\begin{document}
\subject[2110203 - Computer Engineering Mathematics II]
\hwtitle{Optimize 1}{}{Week 6}{6733172621 Patthadon Phengpinij}{ChatGPT (for\,\LaTeX\,styling and grammar checking)}


% ================================================================================ %
\section{Linear Programming (LP)}
% ================================================================================ %


% ================================================================================ %
%                                    Problem 01                                    %
% ================================================================================ %
\begin{problem}
Which of the following shapes are convex sets?
\begin{center}
    \includegraphics[width=0.8\textwidth]{images/problem-01-questions.png}
\end{center}
\end{problem}

\begin{solution} \\
The Definition of Convex Set:

\vspace{2mm}

\par\hspace{5mm} A set \( S \) in a vector space is called convex if, for any two points \( u, v \in S \), the line segment connecting them lies entirely within \( S \).
Mathematically, this means that for any \( \lambda \) such that \( 0 \leq \lambda \leq 1 \), the point \( \lambda u + (1 - \lambda) v \) is also in \( S \).

\vspace{2mm}

\par\hspace{5mm} Now, let's analyze each shape:

\begin{center}
    \includegraphics[width=0.8\textwidth]{images/problem-01-non-convex.png}
\end{center}

As shown in the figure above, shape (2), (3), (5), (7), and (8) are \textbf{not convex sets} because there exist at least one pair of points within each shape such that the line segment connecting them extends outside the shape.

\vspace{3mm}

\par\hspace{5mm} Next, we analyze the shape (1), (4), and (6):
\begin{enumerate}
    \item Shape (1): This shape is a oval, and for any two points within the oval, the line segment connecting them lies entirely within the shape. Therefore, shape (1) is a \textbf{convex set}.
    \item Shape (4): This shape is a intersection of a circle and a halfspace (the region that satisfies a linear inequality, region on one side of a line) which is convex. The intersection of two convex sets is also convex. Therefore, shape (4) is a \textbf{convex set}.
    \item Shape (6): This shape is a parallelogram, which is a polyhedron. Therefore, shape (6) is a \textbf{convex set}.
\end{enumerate}
\end{solution}
% ================================================================================ %


% ================================================================================ %
%                                    Problem 02                                    %
% ================================================================================ %
\begin{problem}
Determine whether each of the following LPs has unique solution, infinite solutions, infeasible, or unbounded.
\end{problem}

% === Problem 2.1. === %
\begin{subproblems_alpha}
    \item
    \[
    \begin{array}{@{\hspace{2mm}} r @{\hspace{2mm}} r @{\hspace{2mm}} c @{\hspace{2mm}} l}
        \text{max } & z & = & x_1 + x_2 \\
        \text{subject to } & 3x_1 + 2x_2 & \leq & 6 \\
        & x_1 + 2x_2 & \leq & 4 \\
        & x_1, x_2 & \geq & 0
    \end{array}
    \]
\end{subproblems_alpha}

\begin{solution}
Frist, we re-write the LP in standard form:
\[
\begin{array}{@{\hspace{2mm}} r @{\hspace{2mm}} r @{\hspace{2mm}} c @{\hspace{2mm}} l}
    \text{max } & z & = & x_1 + x_2 \\
    \text{subject to } & 3x_1 + 2x_2 + s_1 & = & 6 \\
    & x_1 + 2x_2 + s_2 & = & 4 \\
    & x_1, x_2, s_1, s_2 & \geq & 0
\end{array}
\]

Next, we find all basic feasible solutions (BFSs) by setting two variables to zero and solving for the others.
\[
\renewcommand{\arraystretch}{1.2}
\begin{array}{cccc|cc}
    \hline
    x_1 & x_2 & s_1 & s_2 & \text{BFS}? & z \\
    \hline
    \textcolor[HTML]{0000FF}{0} & \textcolor[HTML]{0000FF}{0} & 6 & 4 & \text{Yes} & 0 \\
    \textcolor[HTML]{0000FF}{0} & 3 & \textcolor[HTML]{0000FF}{0} & \textcolor[HTML]{FF0000}{-2} & \textcolor[HTML]{FF0000}{\text{No}} & - \\
    \textcolor[HTML]{0000FF}{0} & 2 & 2 & \textcolor[HTML]{0000FF}{0} & \text{Yes} & 2 \\
    2 & \textcolor[HTML]{0000FF}{0} & \textcolor[HTML]{0000FF}{0} & 2 & \text{Yes} & 2 \\
    4 & \textcolor[HTML]{0000FF}{0} & \textcolor[HTML]{FF0000}{-6} & \textcolor[HTML]{0000FF}{0} & \textcolor[HTML]{FF0000}{\text{No}} & - \\
    1 & \frac{3}{2} & \textcolor[HTML]{0000FF}{0} & \textcolor[HTML]{0000FF}{0} & \text{Yes} & \frac{5}{2} \\
    \hline
\end{array}
\]

Since, there is only one BFS that gives the maximum value of \( z = \frac{5}{2} \) at \( (x_1, x_2) = \paren{ 1, \frac{3}{2} } \), the LP has a \textbf{unique solution}.
\end{solution}
% ==================== %

\newpage

% === Problem 2.2. === %
\begin{subproblems_alpha}[start=2]
    \item
    \[
    \begin{array}{@{\hspace{2mm}} r @{\hspace{2mm}} r @{\hspace{2mm}} c @{\hspace{2mm}} l}
        \text{max } & z & = & \frac{x_1}{2} + x_2 \\
        \text{subject to } & 3x_1 + 2x_2 & \leq & 6 \\
        & x_1 + 2x_2 & \leq & 4 \\
        & x_1, x_2 & \geq & 0
    \end{array}
    \]
\end{subproblems_alpha}

\begin{solution}
Frist, we re-write the LP in standard form:
\[
\begin{array}{@{\hspace{2mm}} r @{\hspace{2mm}} r @{\hspace{2mm}} c @{\hspace{2mm}} l}
    \text{max } & z & = & \frac{x_1}{2} + x_2 \\
    \text{subject to } & 3x_1 + 2x_2 + s_1 & = & 6 \\
    & x_1 + 2x_2 + s_2 & = & 4 \\
    & x_1, x_2, s_1, s_2 & \geq & 0
\end{array}
\]

Next, we find all basic feasible solutions (BFSs) by setting two variables to zero and solving for the others.
\[
\renewcommand{\arraystretch}{1.2}
\begin{array}{cccc|cc}
    \hline
    x_1 & x_2 & s_1 & s_2 & \text{BFS}? & z \\
    \hline
    \textcolor[HTML]{0000FF}{0} & \textcolor[HTML]{0000FF}{0} & 6 & 4 & \text{Yes} & 0 \\
    \textcolor[HTML]{0000FF}{0} & 3 & \textcolor[HTML]{0000FF}{0} & \textcolor[HTML]{FF0000}{-2} & \textcolor[HTML]{FF0000}{\text{No}} & - \\
    \textcolor[HTML]{0000FF}{0} & 2 & 2 & \textcolor[HTML]{0000FF}{0} & \text{Yes} & 2 \\
    2 & \textcolor[HTML]{0000FF}{0} & \textcolor[HTML]{0000FF}{0} & 2 & \text{Yes} & 1 \\
    4 & \textcolor[HTML]{0000FF}{0} & \textcolor[HTML]{FF0000}{-6} & \textcolor[HTML]{0000FF}{0} & \textcolor[HTML]{FF0000}{\text{No}} & - \\
    1 & \frac{3}{2} & \textcolor[HTML]{0000FF}{0} & \textcolor[HTML]{0000FF}{0} & \text{Yes} & 2 \\
    \hline
\end{array}
\]

Since, there are two BFSs that give the maximum value of \( z = 2 \) at \( (x_1, x_2) = \paren{ 0, 2 } \) and \( (x_1, x_2) = \paren{ 1, \frac{3}{2} } \), the LP has \textbf{infinite solutions}.
\end{solution}
% ==================== %


% === Problem 2.3. === %
\begin{subproblems_alpha}[start=3]
    \item
    \[
    \begin{array}{@{\hspace{2mm}} r @{\hspace{2mm}} r @{\hspace{2mm}} c @{\hspace{2mm}} l}
        \text{max } & z & = & x_1 + x_2 \\
        \text{subject to } & 3x_1 + 2x_2 & \geq & 6 \\
        & x_1 + 2x_2 & \geq & 4 \\
        & x_1, x_2 & \geq & 0
    \end{array}
    \]
\end{subproblems_alpha}

\begin{solution}
Consider the constraints, since both inequalities are of the form \( \boxed{\geq} \) and \( \boxed{x_1, x_2 \geq 0} \) with \textbf{positive coefficients} on both variables,
the feasible region is \underline{unbounded} in the direction of increasing \( x_1 \) and \( x_2 \).

\vspace{2mm}

\par\hspace{5mm} To see this more clearly, we can easily substitute \( x_1 \to \infty \) and \( x_2 \to \infty \) into the LP:
\[
\begin{array}{@{\hspace{2mm}} r @{\hspace{2mm}} r @{\hspace{2mm}} c @{\hspace{2mm}} l}
    \text{max } & z & = & \infty + \infty = \infty \\
    \text{subject to } & 3(\infty) + 2(\infty) & \geq & 6 \\
    & (\infty) + 2(\infty) & \geq & 4 \\
    & \infty, \infty & \geq & 0
\end{array}
\]

From the above substitution, we can see that both constraints are satisfied as \( \infty \geq 6 \) and \( \infty \geq 4 \), confirming that \( x_1 \to \infty \) and \( x_2 \to \infty \) are indeed \underline{feasible solutions}.

\vspace{2mm}

\par\hspace{5mm} Therefore, the LP is \textbf{unbounded} because we can increase \( x_1 \) and \( x_2 \) indefinitely while still satisfying the constraints, leading to an infinitely large value of \( z \).
\end{solution}
% ==================== %
% ================================================================================ %

\newpage

% ================================================================================ %
%                                    Problem 03                                    %
% ================================================================================ %
\begin{problem}
Convert the following LP to standard form and find all BFSs.
\[
\begin{array}{@{\hspace{2mm}} r @{\hspace{2mm}} r @{\hspace{2mm}} c @{\hspace{2mm}} l}
    \text{max } & z & = & x_1 - x_2 \\
    \text{subject to } & x_1 + x_2 & \leq & 3 \\
    & x_1 + 3x_2 & \leq & 4 \\
    & 2x_1 - x_2 & \geq & 1 \\
    & x_1, x_2 & \geq & 0
\end{array}
\]
\end{problem}

\begin{solution}
To convert the given LP to standard form, we need to express all constraints as equalities and ensure all variables are non-negative.

\vspace{2mm}

\par\hspace{5mm} We introduce slack variables \( s_1 \) and \( s_2 \) for the first two inequalities and a excess variable \( e_1 \) for the third inequality. The standard form of the LP is:
\[
\begin{array}{@{\hspace{2mm}} r @{\hspace{2mm}} r @{\hspace{2mm}} c @{\hspace{2mm}} l}
    \text{max } & z & = & x_1 - x_2 \\
    \text{subject to } & x_1 + x_2 + s_1 & = & 3 \\
    & x_1 + 3x_2 + s_2 & = & 4 \\
    & 2x_1 - x_2 - e_1 & = & 1 \\
    & x_1, x_2, s_1, s_2, e_1 & \geq & 0
\end{array}
\]

\par\hspace{5mm} Next, we find all basic feasible solutions (BFSs) by setting two variables to zero and solving for the others.
\[
\renewcommand{\arraystretch}{1.2}
\begin{array}{ccccc|cc}
    \hline
    x_1 & x_2 & s_1 & s_2 & e_1 & \text{BFS}? & z \\
    \hline
    \textcolor[HTML]{0000FF}{0} & \textcolor[HTML]{0000FF}{0} & 3 & 4 & \textcolor[HTML]{FF0000}{-1} & \textcolor[HTML]{FF0000}{\text{No}} & - \\
    \textcolor[HTML]{0000FF}{0} & 3 & \textcolor[HTML]{0000FF}{0} & \textcolor[HTML]{FF0000}{-5} & \textcolor[HTML]{FF0000}{-4} & \textcolor[HTML]{FF0000}{\text{No}} & - \\
    \textcolor[HTML]{0000FF}{0} & \frac{4}{3} & \frac{5}{3} & \textcolor[HTML]{0000FF}{0} & \textcolor[HTML]{FF0000}{-\frac{7}{3}} & \textcolor[HTML]{FF0000}{\text{No}} & - \\
    \textcolor[HTML]{0000FF}{0} & \textcolor[HTML]{FF0000}{-1} & 4 & 7 & \textcolor[HTML]{0000FF}{0} & \textcolor[HTML]{FF0000}{\text{No}} & - \\
    3 & \textcolor[HTML]{0000FF}{0} & \textcolor[HTML]{0000FF}{0} & 1 & 5 & \text{Yes} & 3 \\
    4 & \textcolor[HTML]{0000FF}{0} & \textcolor[HTML]{FF0000}{-1} & \textcolor[HTML]{0000FF}{0} & 7 & \textcolor[HTML]{FF0000}{\text{No}} & - \\
    \frac{1}{2} & \textcolor[HTML]{0000FF}{0} & \frac{5}{2} & \frac{7}{2} & \textcolor[HTML]{0000FF}{0} & \text{Yes} & \frac{1}{2} \\
    \frac{5}{2} & \frac{1}{2} & \textcolor[HTML]{0000FF}{0} & \textcolor[HTML]{0000FF}{0} & \frac{7}{2} & \text{Yes} & 2 \\
    \frac{4}{3} & \frac{5}{3} & \textcolor[HTML]{0000FF}{0} & \textcolor[HTML]{FF0000}{-\frac{7}{3}} & \textcolor[HTML]{0000FF}{0} & \textcolor[HTML]{FF0000}{\text{No}} & - \\
    1 & 1 & 1 & \textcolor[HTML]{0000FF}{0} & \textcolor[HTML]{0000FF}{0} & \text{Yes} & 0 \\
    \hline
\end{array}
\]

From the above table, the BFSs are:
\begin{itemize}
    \item \( (x_1, x_2, s_1, s_2, e_1) = (3, 0, 0, 1, 5) \) with \( z = 3 \)
    \item \( (x_1, x_2, s_1, s_2, e_1) = \paren{ \frac{1}{2}, 0, \frac{5}{2}, \frac{7}{2}, 0 } \) with \( z = \frac{1}{2} \)
    \item \( (x_1, x_2, s_1, s_2, e_1) = \paren{ \frac{5}{2}, \frac{1}{2}, 0, 0, \frac{7}{2} } \) with \( z = 2 \)
    \item \( (x_1, x_2, s_1, s_2, e_1) = (1, 1, 1, 0, 0) \) with \( z = 0 \)
\end{itemize}

In term of \( (x_1, x_2) \), the BFSs are:
\begin{itemize}
    \item \( (3, 0) \) with \( z = 3 \)
    \item \( \paren{ \frac{1}{2}, 0 } \) with \( z = \frac{1}{2} \)
    \item \( \paren{ \frac{5}{2}, \frac{1}{2} } \) with \( z = 2 \)
    \item \( (1, 1) \) with \( z = 0 \)
\end{itemize}
\end{solution}
% ================================================================================ %

\newpage

% ================================================================================ %
%                                    Problem 04                                    %
% ================================================================================ %
\begin{tosubmit}
\problem[4] Formulate the following problem into LP.
Then, convert it to standard form and find all BFSs. Is the optimal BFS unique?

\vspace{3mm}

In the 2036 Summer Olympics, Thailand wins a total of 2 gold, 6 silver, and 8 bronze medals.
Among these, 1 gold, 2 silver, and 3 bronze medals come from equestrian events.
The government wants to award total money of exactly 60 million baht to the medalists such that people winning the same type of medal will receive the same amount of money.
Also, a gold medalist should receive at least 2 million baht more than a silver medalist, and a silver medalist should receive at least 1 million baht more than a bronze medalist.
Find the minimum possible amount of money that all equestrians will receive in total.

\vspace{3mm}

\textit{Remark:} It is possible that some people may receive no money.

\vspace{3mm}

\par\noindent\submitsolution
Let \( x_1 \), \( x_2 \), and \( x_3 \) be the amount of money (in million baht) awarded to each gold, silver, and bronze medalist, respectively.
Thus,
\begin{itemize}
    \item The total amount of money awarded to all medalists is given by:
    \[ 2x_1 + 6x_2 + 8x_3 = 60 \]

    \item The constraints for the amounts awarded to each type of medalist are:
    \begin{align*}
        x_1 &\geq x_2 + 2 \\
        x_2 &\geq x_3 + 1
    \end{align*}

    \item The objective is to minimize the total amount awarded to equestrian medalists:
    \[ \text{minimize } z = x_1 + 2x_2 + 3x_3 \]

    \item The non-negativity constraints are:
    \[ x_1, x_2, x_3 \geq 0 \]
\end{itemize}

\vspace{2mm}

Therefore, the LP formulation is:
\[
\begin{array}{@{\hspace{2mm}} r @{\hspace{2mm}} r @{\hspace{2mm}} c @{\hspace{2mm}} l}
    \text{min } & z & = & x_1 + 2x_2 + 3x_3 \\
    \text{subject to } & 2x_1 + 6x_2 + 8x_3 & = & 60 \\
    & x_1 - x_2 & \geq & 2 \\
    & x_2 - x_3 & \geq & 1 \\
    & x_1, x_2, x_3 & \geq & 0
\end{array}
\]

Re-write the LP in standard form by introducing excess variables \( e_1 \) and \( e_2 \):
\[
\begin{array}{@{\hspace{2mm}} r @{\hspace{2mm}} r @{\hspace{2mm}} c @{\hspace{2mm}} l}
    \text{max } & z' & = & -x_1 - 2x_2 - 3x_3 \\
    \text{subject to } & 2x_1 + 6x_2 + 8x_3 & = & 60 \\
    & x_1 - x_2 - e_1 & = & 2 \\
    & x_2 - x_3 - e_2 & = & 1 \\
    & x_1, x_2, x_3, e_1, e_2 & \geq & 0
\end{array}
\]

\newpage

Next, we find all basic feasible solutions (BFSs) by setting two variables to zero and solving for the others.
\[
\renewcommand{\arraystretch}{1.2}
\begin{array}{ccccc|cc}
    \hline
    x_1 & x_2 & x_3 & e_1 & e_2 & \text{BFS}? & z \\
    \hline
    \textcolor[HTML]{0000FF}{0} & \textcolor[HTML]{0000FF}{0} & \frac{15}{2} & \textcolor[HTML]{FF0000}{-2} & \textcolor[HTML]{FF0000}{-\frac{17}{2}} & \textcolor[HTML]{FF0000}{\text{No}} & - \\
    \textcolor[HTML]{0000FF}{0} & 10 & \textcolor[HTML]{0000FF}{0} & \textcolor[HTML]{FF0000}{-12} & 9 & \textcolor[HTML]{FF0000}{\text{No}} & - \\
    \textcolor[HTML]{0000FF}{0} & \textcolor[HTML]{FF0000}{-2} & 9 & \textcolor[HTML]{0000FF}{0} & \textcolor[HTML]{FF0000}{-12} & \textcolor[HTML]{FF0000}{\text{No}} & - \\
    \textcolor[HTML]{0000FF}{0} & \frac{34}{7} & \frac{27}{7} & \textcolor[HTML]{FF0000}{-\frac{48}{7}} & \textcolor[HTML]{0000FF}{0} & \textcolor[HTML]{FF0000}{\text{No}} & - \\
    30 & \textcolor[HTML]{0000FF}{0} & \textcolor[HTML]{0000FF}{0} & 28 & \textcolor[HTML]{FF0000}{-1} & \textcolor[HTML]{FF0000}{\text{No}} & - \\
    2 & \textcolor[HTML]{0000FF}{0} & 7 & \textcolor[HTML]{0000FF}{0} & \textcolor[HTML]{FF0000}{-8} & \textcolor[HTML]{FF0000}{\text{No}} & - \\
    34 & \textcolor[HTML]{0000FF}{0} & \textcolor[HTML]{FF0000}{-1} & 32 & \textcolor[HTML]{0000FF}{0} & \textcolor[HTML]{FF0000}{\text{No}} & - \\
    9 & 7 & \textcolor[HTML]{0000FF}{0} & \textcolor[HTML]{0000FF}{0} & 6 & \text{Yes} & -23 \\
    27 & 1 & \textcolor[HTML]{0000FF}{0} & 24 & \textcolor[HTML]{0000FF}{0} & \text{Yes} & -29 \\
    6 & 4 & 3 & \textcolor[HTML]{0000FF}{0} & \textcolor[HTML]{0000FF}{0} & \text{Yes} & -23 \\
    \hline
\end{array}
\]

From the above table, the BFSs are:
\begin{itemize}
    \item \( (x_1, x_2, x_3, e_1, e_2) = (9, 7, 0, 0, 6) \) with \( z = 23 \)
    \item \( (x_1, x_2, x_3, e_1, e_2) = (27, 1, 0, 24, 0) \) with \( z = 29 \)
    \item \( (x_1, x_2, x_3, e_1, e_2) = (6, 4, 3, 0, 0) \) with \( z = 23 \)
\end{itemize}

We want to minimize \( z \) (maximize \( -z \)), so the optimal value is \[ \boxed{z = 23} \] at two BFSs: \( (9, 7, 0) \) and \( (6, 4, 3) \) (the optimal BFS is not unique, there are \textbf{infinite solutions}).
\end{tosubmit}
% ================================================================================ %


\end{document}