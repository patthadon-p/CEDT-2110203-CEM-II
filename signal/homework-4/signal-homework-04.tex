\documentclass[a4paper, 10pt]{article}
\usepackage{../../CEDT-Homework-style}

\usepackage{amsmath}
\allowdisplaybreaks

\setlength{\headheight}{14.49998pt}

\begin{document}
\subject[2110203 - Computer Engineering Mathematics II]
\hwtitle{Signal 4}{}{Week 4}{6733172621 Patthadon Phengpinij}{ChatGPT (for\,\LaTeX\,styling and grammar checking)}


% ================================================================================ %
\section{Continuous-Time Fourier Transform (CTFT)}
% ================================================================================ %



% ================================================================================ %
%                                    Problem 01                                    %
% ================================================================================ %
\begin{problem}
    Find the Fourier transform of the following signals in terms of \( X(j\omega) \),
    the Fourier transform of \( x(t) \; \paren{ \mathscr{F} \set{ x(t) } = X(j\omega) } \)
\end{problem}

% === 1.1 === %
\begin{subproblems}[start=1]
    \item \( x(-t) \)
\end{subproblems}

\begin{solution}
Using the Time-scaling property of the Fourier transform, we have:
\[
    \textbf{Time-scaling: } \mathscr{F} \set{ x(at) } = \frac{1}{|a|} X\paren{ \frac{j\omega}{a} }
\]

Substituting \( a = -1 \) into the time-scaling property, we get:
\begin{align*}  
    \mathscr{F} \set{ x(-t) } &= \frac{1}{|-1|} X\paren{ \frac{j\omega}{-1} }
\end{align*}

Therefore, we can express the Fourier transform of \( x(-t) \) as:
\[ \boxed{
    \mathscr{F} \set{ x(-t) } = X(-j\omega)
} \]
\end{solution}
% =========== %


% === 1.2 === %
\begin{subproblems}[resume]
    \item \( x_e(t) = \frac{x(t) + x(-t)}{2} \)
\end{subproblems}

\begin{solution}
Using the Time-scaling and Linearity properties of the Fourier transform, we have:
\[
    \textbf{Time-scaling: } \mathscr{F} \set{ x(at) } = \frac{1}{|a|} X\paren{ \frac{j\omega}{a} }
\]
\[
    \textbf{Linearity: } \mathscr{F} \set{ ax_1(t) + bx_2(t) } = aX_1(j\omega) + bX_2(j\omega)
\]

Considering the signal \( x_e(t) = \frac{x(t) + x(-t)}{2} \), we can find its Fourier transform as follows:
\begin{align*}  
    \mathscr{F} \set{ x_e(t) } &= \mathscr{F} \set{ \frac{x(t) + x(-t)}{2} } \\
    &= \frac{1}{2} \paren{ \mathscr{F} \set{ x(t) } + \mathscr{F} \set{ x(-t) } } \\
    \mathscr{F} \set{ x_e(t) } &= \frac{1}{2} \paren{ X(j\omega) + \frac{1}{|-1|} X\paren{ \frac{j\omega}{-1} } }
\end{align*}

Therefore, we can express the Fourier transform of \( x_e(t) \) as:
\[ \boxed{
    \mathscr{F} \set{ x_e(t) } = \frac{X(j\omega) + X(-j\omega)}{2}
} \]
\end{solution}
% =========== %

\newpage

% === 1.3 === %
\begin{subproblems}[resume]
    \item \( x_o(t) = \frac{x(t) - x(-t)}{2} \)
\end{subproblems}

\begin{solution}
Using the Time-scaling and Linearity properties of the Fourier transform, we have:
\[
    \textbf{Time-scaling: } \mathscr{F} \set{ x(at) } = \frac{1}{|a|} X\paren{ \frac{j\omega}{a} }
\]
\[
    \textbf{Linearity: } \mathscr{F} \set{ ax_1(t) + bx_2(t) } = aX_1(j\omega) + bX_2(j\omega)
\]

Considering the signal \( x_o(t) = \frac{x(t) - x(-t)}{2} \), we can find its Fourier transform as follows:
\begin{align*}  
    \mathscr{F} \set{ x_o(t) } &= \mathscr{F} \set{ \frac{x(t) - x(-t)}{2} } \\
    &= \frac{1}{2} \paren{ \mathscr{F} \set{ x(t) } - \mathscr{F} \set{ x(-t) } } \\
    \mathscr{F} \set{ x_o(t) } &= \frac{1}{2} \paren{ X(j\omega) - \frac{1}{|-1|} X\paren{ \frac{j\omega}{-1} } }
\end{align*}

Therefore, we can express the Fourier transform of \( x_o(t) \) as:
\[ \boxed{
    \mathscr{F} \set{ x_o(t) } = \frac{X(j\omega) - X(-j\omega)}{2}
} \]
\end{solution}
% =========== %
% ================================================================================ %


% ================================================================================ %
%                                    Problem 02                                    %
% ================================================================================ %
\begin{problem}
    Let \( \mathscr{F} \set{ x(t) } = X(j\omega) = \textrm{rect} \paren{ (\omega-1)/2 } \). Find Fourier tranform of
\end{problem}

% === 2.1 === %
\begin{tosubmit}
\begin{subproblems}[start=1]
    \item \( x(-2t+4) \)
\end{subproblems}

\par\noindent\submitsolution
From the Time-scaling and Time-shifting properties of the Fourier transform, we have:
\[
    \textbf{Time-scaling: } \mathscr{F} \set{ x(at) } = \frac{1}{|a|} X\paren{ \frac{j\omega}{a} }
\]
\[
    \textbf{Time-shifting: } \mathscr{F} \set{ x(t - t_0) } = e^{-j\omega t_0} X(j\omega)
\]

Combining these two properties, we can find the Fourier transform of \( x(at - b) \).
\[
    \mathscr{F} \set{ x(at - b) } = \mathscr{F} \set{ x \paren{ a \paren{ t - \frac{b}{a} } } } = \frac{1}{|a|} e^{-j\omega \frac{b}{a}} X\paren{ \frac{j\omega}{a} }
\]

To find the Fourier transform of \( x(-2t + 4) \), we have \( a = -2 \) and \( b = -4 \). Applying the combined properties and substituting \( \mathscr{F} \set{ x(t) } \), we get:
\begin{align*}  
    \mathscr{F} \set{ x(-2t + 4) } &= \frac{1}{|a|} e^{-j\omega \frac{b}{a}} X\paren{ \frac{j\omega}{a} } \\
    &= \frac{1}{|-2|} e^{-j\omega \frac{-4}{-2}} X\paren{ \frac{j\omega}{-2} } \\
    &= \frac{1}{2} e^{-2j\omega} \textrm{rect} \paren{ \frac{\frac{\omega}{-2} - 1}{2} } \\
    \mathscr{F} \set{ x(-2t + 4) } &= \frac{1}{2} e^{-2j\omega} \textrm{rect} \paren{ \frac{-\omega - 2}{4} }
\end{align*}

Because, \( \textrm{rect} \) is an even function, we can express the Fourier transform of \( x(-2t + 4) \) as:
\[ \boxed{
    \mathscr{F} \set{ x(-2t + 4) } = \frac{1}{2} e^{-2j\omega} \textrm{rect} \paren{ \frac{\omega + 2}{4} }
} \]
\end{tosubmit}
% =========== %

\newpage

% === 2.2 === %
\begin{tosubmit}
\begin{subproblems}[start=2]
    \item \( (t-1)x(t-1) \)
\end{subproblems}

\par\noindent\submitsolution
Using the Time-shifting property of the Fourier transform, we have:
\[
    \textbf{Time-shifting: } \mathscr{F} \set{ x(t - t_0) } = e^{-j\omega t_0} X(j\omega)
\]

First, define a new signal \( y(t) = tx(t) \). Then, we can express \( (t-1)x(t-1) \) as:
\[
    (t-1)x(t-1) = y(t-1)
\]

Now, applying the Time-shifting property to \( y(t-1) \), we get:
\begin{align*}  
    \mathscr{F} \set{ y(t-1) } &= e^{-j\omega \cdot 1} Y(j\omega) \\
    &= e^{-j\omega} \mathscr{F} \set{ y(t) } \\
    \mathscr{F} \set{ y(t-1) } &= e^{-j\omega} \mathscr{F} \set{ tx(t) }
\end{align*}

Next, consider the differentiation of \( \mathscr{F} \set{ x(t) } \):
\begin{align*}
    \frac{d}{d\omega} \mathscr{F} \set{ x(t) } &= \frac{d}{d\omega} \int_{-\infty}^{\infty} x(t) e^{-j\omega t} \,dt \\
    &= \int_{-\infty}^{\infty} x(t) \frac{d}{d\omega} \paren{ e^{-j\omega t} } \,dt \\
    &= \int_{-\infty}^{\infty} x(t) \paren{ -jt e^{-j\omega t} } \,dt \\
    &= -j \int_{-\infty}^{\infty} t x(t) e^{-j\omega t} \,dt \\
    \frac{d}{d\omega} \mathscr{F} \set{ x(t) }  &= -j \mathscr{F} \set{ t x(t) } \\
    \mathscr{F} \set{ t x(t) }  &= j \frac{d}{d\omega} X(j\omega)
\end{align*}

Substituting \( \mathscr{F} \set{ x(t) } \) into the equation, we get:
\begin{align*}  
    \mathscr{F} \set{ tx(t) } &= j \frac{d}{d\omega} X(j\omega) \\
    &= j \frac{d}{d\omega} \paren{ \textrm{rect} \paren{ \frac{\omega - 1}{2} } } \\
    &= j \frac{d}{d\omega} \paren{ u\paren{ \omega } - u\paren{ \omega - 2 } } \\
    \mathscr{F} \set{ tx(t) } &= j \paren{ \delta(\omega) - \delta(\omega - 2) }
\end{align*}

Therefore, we can express the Fourier transform of \( (t-1)x(t-1) \) as:
\[ \boxed{
    \mathscr{F} \set{ (t-1)x(t-1) } = j e^{-j\omega} \paren{ \delta(\omega) - \delta(\omega - 2) }
} \]
\end{tosubmit}
% =========== %

\newpage

% === 2.3 === %
\begin{tosubmit}
\begin{subproblems}[start=3]
    \item \( t \frac{d x(t)}{dt} \)
\end{subproblems}

\par\noindent\submitsolution
Using the Differentiation in time property of the Fourier transform, we have:
\[
    \textbf{Differentiation in time: } \mathscr{F} \set{ \frac{d x(t)}{dt} } = j\omega X(j\omega)
\]

And, using the Differentiation in frequency property (proved in the previous problem) of the Fourier transform, we have:
\[
    \textbf{Differentiation in frequency: } \mathscr{F} \set{ t x(t) } = j \frac{d}{d\omega} X(j\omega)
\]

First, define a new signal \( y(t) = \frac{d x(t)}{dt} \). Then, we can express \( t \frac{d x(t)}{dt} \) as:
\[
    t \frac{d x(t)}{dt} = t y(t)
\]

Now, applying the Differentiation in frequency property to \( t y(t) \), we get:
\[
    \mathscr{F} \set{ t y(t) } = j \frac{d}{d\omega} Y(j\omega) = j \frac{d}{d\omega} \mathscr{F} \set{ y(t) } = j \frac{d}{d\omega} \mathscr{F} \set{ \frac{d x(t)}{dt} }
\]

Next, substituting the Differentiation in time property into the equation, we get:
\begin{align*}  
    \frac{d}{d\omega} \mathscr{F} \set{ \frac{d x(t)}{dt} } &= \frac{d}{d\omega} \paren{ j\omega X(j\omega) } \\
    &= \frac{d}{d\omega} \paren{ j\omega \cdot \textrm{rect} \paren{ \frac{\omega - 1}{2} } } \\
    &= \frac{d}{d\omega} \paren{ j\omega \paren{ u(\omega) - u(\omega - 2) } } \\
    &= j\omega \frac{d}{d\omega} \paren{ u(\omega) - u(\omega - 2) } + \paren{ u(\omega) - u(\omega - 2) } \frac{d}{d\omega} \paren{ j\omega } \\
    &= j\omega \paren{ \delta(\omega) - \delta(\omega - 2) } + \textrm{rect} \paren{ \frac{\omega - 1}{2} } \cdot (j) \\
    &= j\omega \delta(\omega) - j\omega \delta(\omega - 2) + j \textrm{rect} \paren{ \frac{\omega - 1}{2} } \\
    &= 0 - j(2) \delta(\omega - 2) + j \textrm{rect} \paren{ \frac{\omega - 1}{2} } \\
    \frac{d}{d\omega} \mathscr{F} \set{ \frac{d x(t)}{dt} } &= - j(2) \delta(\omega - 2) + j \textrm{rect} \paren{ \frac{\omega - 1}{2} }
\end{align*}

Thus, substituting back, we have:
\begin{align*}  
    \mathscr{F} \set{ ty(t) } &= j \frac{d}{d\omega} \mathscr{F} \set{ \frac{d x(t)}{dt} } \\
    &= j \cdot \sqbracket{ - j(2) \delta(\omega - 2) + j \textrm{rect} \paren{ \frac{\omega - 1}{2} } } \\
    \mathscr{F} \set{ ty(t) } &= 2\delta(\omega - 2) - \textrm{rect} \paren{ \frac{\omega - 1}{2} }
\end{align*}

Therefore, we can express the Fourier transform of \( t \frac{d x(t)}{dt} \) as:
\[ \boxed{
    \mathscr{F} \set{ t \frac{d x(t)}{dt} } = 2\delta(\omega - 2) - \textrm{rect} \paren{ \frac{\omega - 1}{2} }
} \]
\end{tosubmit}
% =========== %

\newpage

% === 2.4 === %
\begin{tosubmit}
\begin{subproblems}[start=4]
    \item \( x(2t-1) e^{-j2t} \)
\end{subproblems}

\par\noindent\submitsolution
Using the Time-scaling, Time-shifting, and Frequency-shifting properties of the Fourier transform, we have:
\[
    \textbf{Time-scaling + Time-shifting: } \mathscr{F} \set{ x(at - b) } = \frac{1}{|a|} e^{-j\omega \frac{b}{a}} X\paren{ \frac{j\omega}{a} }
\]
\[
    \textbf{Frequency-shifting: } \mathscr{F} \set{ x(t) e^{j\omega_0 t} } = X(j(\omega - \omega_0))
\]

Define a new signal \( y(t) = x(2t - 1) \). Then, we can express \( x(2t - 1) e^{-j2t} \) as:
\begin{align*}
    Y(j\omega) &= \mathscr{F} \set{ y(t) } \\
    &= \mathscr{F} \set{ x(2t - 1) } \\
    &= \frac{1}{|2|} e^{-j\omega \frac{1}{2}} X\paren{ \frac{j\omega}{2} } \\
    Y(j\omega) &= \frac{1}{2} e^{-j\frac{\omega}{2}} X\paren{ \frac{j\omega}{2} }
\end{align*}

Now, applying the Frequency-shifting property to \( y(t) e^{-j2t} \), we get:
\begin{align*}  
    \mathscr{F} \set{ y(t) e^{-j2t} } &= \mathscr{F} \set{ y(t) e^{j(-2)t} } \\
    &= Y(j(\omega - (-2))) \\
    &= Y(j(\omega + 2)) \\
    \mathscr{F} \set{ y(t) e^{-j2t} } &= \frac{1}{2} e^{-j\frac{\omega + 2}{2}} X\paren{ \frac{j(\omega + 2)}{2} }
\end{align*}

Then, substituting \( \mathscr{F} \set{ x(t) } \) into the equation, we get:
\begin{align*}  
    \mathscr{F} \set{ y(t) e^{-j2t} } &= \frac{1}{2} e^{-j\frac{\omega + 2}{2}} X\paren{ \frac{j(\omega + 2)}{2} } \\
    &= \frac{1}{2} e^{-j\frac{\omega + 2}{2}} \textrm{rect} \paren{ \frac{\frac{\omega + 2}{2} - 1}{2} } \\
    &= \frac{1}{2} e^{-j\frac{\omega + 2}{2}} \textrm{rect} \paren{ \frac{\omega + 2 - 2}{4} } \\
    \mathscr{F} \set{ y(t) e^{-j2t} } &= \frac{1}{2} e^{-j\frac{\omega + 2}{2}} \textrm{rect} \paren{ \frac{\omega}{4} }
\end{align*}

Therefore, we can express the Fourier transform of \( x(2t - 1) e^{-j2t} \) as:
\[ \boxed{
    \mathscr{F} \set{ x(2t - 1) e^{-j2t} } = \frac{1}{2} e^{-j\frac{\omega + 2}{2}} \textrm{rect} \paren{ \frac{\omega}{4} }
} \]
\end{tosubmit}
% =========== %

\newpage

% === 2.5 === %
\begin{tosubmit}
\begin{subproblems}[start=5]
    \item \( x(t) * x(t-1) \)
\end{subproblems}

\par\noindent\submitsolution
Using the Convolution property and Time-shifting property of the Fourier transform, we have:
\[
    \textbf{Convolution: } \mathscr{F} \set{ x_1(t) * x_2(t) } = X_1(j\omega) \cdot X_2(j\omega)
\]
\[
    \textbf{Time-shifting: } \mathscr{F} \set{ x(t - t_0) } = e^{-j\omega t_0} X(j\omega)
\]

First, define a new signal \( y(t) = x(t - 1) \). Then, we can express \( x(t) * x(t - 1) \) as:
\[
    x(t) * x(t - 1) = x(t) * y(t)
\]

Next, substituting the Time-shifting property into the equation, we get:
\begin{align*}  
    Y(j\omega) &= \mathscr{F} \set{ y(t) } \\
    &= \mathscr{F} \set{ x(t - 1) } \\
    &= e^{-j\omega (1)} X(j\omega) \\
    Y(j\omega) &= e^{-j\omega} X(j\omega)
\end{align*}

Now, applying the Convolution property to \( x(t) * y(t) \), we get:
\begin{align*}  
    \mathscr{F} \set{ x(t) * y(t) } &= \mathscr{F} \set{ x(t) } \cdot \mathscr{F} \set{ y(t) } \\
    &= X(j\omega) \cdot \mathscr{F} \set{ x(t - 1) } \\
    \mathscr{F} \set{ x(t) * y(t) } &= X(j\omega) \cdot e^{-j\omega (1)} X(j\omega)
\end{align*}

Lastly, substituting \( \mathscr{F} \set{ x(t) } = \textrm{rect} \paren{ (\omega-1)/2 } \) back into the equation, we get:

Therefore, we can express the Fourier transform of \( x(t) * x(t - 1) \) as:
\[ \boxed{
    \mathscr{F} \set{ x(t) * x(t - 1) } = e^{-j\omega} \textrm{rect}^2 \paren{ \frac{\omega - 1}{2} }
} \]
\end{tosubmit}
% ================================================================================ %

\newpage

% ================================================================================ %
%                                    Problem 03                                    %
% ================================================================================ %
\begin{problem}
\end{problem}

% === 3.1 === %
\begin{subproblems}[start=1]
    \item Proof that \( \mathscr{F} \set{ e^{-|t|} } = \mathscr{F} \set{ exp(-|t|) } = \frac{2}{\omega^2+1} \)
\end{subproblems}

\begin{solution}
Using the definition of the Continuous-Time Fourier Transform (CTFT), we have:
\[
    \mathscr{F} \set{ x(t) } = X(j\omega) = \int_{-\infty}^{\infty} x(t) e^{-j\omega t} \,dt
\]

Substituting \( x(t) = e^{-|t|} \) into the CTFT definition, we get:
\begin{align*}  
    X(j\omega) &= \int_{-\infty}^{\infty} e^{-|t|} e^{-j\omega t} \,dt \\
    &= \int_{-\infty}^{0} e^{t} e^{-j\omega t} \,dt + \int_{0}^{\infty} e^{-t} e^{-j\omega t} \,dt \\
    &= \int_{-\infty}^{0} e^{(1 - j\omega) t} \,dt + \int_{0}^{\infty} e^{-(1 + j\omega) t} \,dt \\
    &= \sqbracket{ \frac{e^{(1 - j\omega) t}}{1 - j\omega} }_{-\infty}^{0} + \sqbracket{ \frac{-e^{-(1 + j\omega) t}}{1 + j\omega} }_{0}^{\infty} \\
    &= \frac{1}{1 - j\omega} + \frac{1}{1 + j\omega} \\
    &= \frac{(1 + j\omega) + (1 - j\omega)}{(1 - j\omega)(1 + j\omega)} \\
    X(j\omega) &= \frac{2}{1 + \omega^2}
\end{align*}

Therefore, we have proven that:
\[ \boxed{
    \mathscr{F} \set{ e^{-|t|} } = \frac{2}{\omega^2 + 1}
} \qed. \]
\end{solution}
% =========== %


% === 3.2 === %
\begin{subproblems}[start=2]
    \item Using the outcome obtained in Problem 3.1, Find the Fourier Transform of the given equation.
\end{subproblems}

% --- 3.2.1 --- %
\begin{subsubproblems}[start=1]
    \item \( \frac{d}{dt} (e^{-|t|}) \)
\end{subsubproblems}

\begin{solution}
Using the Differentiation in time property of the Fourier transform, we have:
\[
    \textbf{Differentiation in time: } \mathscr{F} \set{ \frac{d x(t)}{dt} } = j\omega X(j\omega)
\]

Define a new signal \( y(t) = e^{-|t|} \), applying the Differentiation in time property to \( \frac{d y(t)}{dt} \), we get:
\begin{align*}  
    \mathscr{F} \set{ \frac{d y(t)}{dt} } &= j\omega Y(j\omega) \\
    &= j\omega \mathscr{F} \set{ y(t) } \\
    \mathscr{F} \set{ \frac{d y(t)}{dt} } &= j\omega \mathscr{F} \set{ e^{-|t|} }
\end{align*}

Substituting the result from Problem 3.1 into the equation, we get:
\[ \boxed{
    \mathscr{F} \set{ \frac{d}{dt} (e^{-|t|}) } = \frac{2j\omega}{\omega^2 + 1}
} \]
\end{solution}
% ------------- %

\newpage

% --- 3.2.2 --- %
\begin{subsubproblems}[start=2]
    \item \( \exp(3jt-|2t+2|) \)
\end{subsubproblems}

\begin{solution}
First, define a new signal \( x(t) = e^{-|t|} \) and \( y(t) = x(2t + 2) = e^{-|2t + 2|} \).

\par\noindent Then, we can express \( \exp(3jt - |2t + 2|) \) as:
\[
    \exp(3jt - |2t + 2|) = y(t) e^{j3t}
\]

Using the Time-scaling and Time-shifting property, we have:
\[
    \textbf{Time-scaling + Time-shifting: } \mathscr{F} \set{ x(at - b) } = \frac{1}{|a|} e^{-j\omega \frac{b}{a}} X\paren{ \frac{j\omega}{a} }
\]

Substituting this property into the equation, we get:
\begin{align*}
    Y(j\omega) &= \mathscr{F} \set{ y(t) } \\
    &= \mathscr{F} \set{ x(2t + 2) } \\
    &= \frac{1}{|2|} e^{-j\omega \frac{2}{2}} X\paren{ \frac{j\omega}{2} } \\
    Y(j\omega) &= \frac{1}{2} e^{-j\omega} X\paren{ \frac{j\omega}{2} }
\end{align*}

Now, applying the Frequency-shifting property
\[
    \textbf{Frequency-shifting: } \mathscr{F} \set{ x(t) e^{j\omega_0 t} } = X(j(\omega - \omega_0))
\]

to \( y(t) e^{j3t} \), we get:
\[ 
    \mathscr{F} \set{ y(t) e^{j3t} } = Y(j(\omega - 3))
\]

Then, substituting back and use the result from Problem 3.1, we have:
\begin{align*}  
    \mathscr{F} \set{ y(t) e^{j3t} } &= Y(j(\omega - 3)) \\
    &= \frac{1}{2} e^{-j(\omega - 3)} X\paren{ \frac{j(\omega - 3)}{2} } \\
    &= \frac{1}{2} e^{-j(\omega - 3)} \cdot \frac{2}{\paren{ \frac{\omega - 3}{2} }^2 + 1} \\
    &= \frac{e^{-j(\omega - 3)}}{\frac{(\omega - 3)^2}{4} + 1} \\
    \mathscr{F} \set{ y(t) e^{j3t} } &= \frac{4 e^{-j(\omega - 3)}}{(\omega - 3)^2 + 4}
\end{align*}

Therefore, we can express the Fourier transform of \( \exp(3jt - |2t + 2|) \) as:
\[ \boxed{
    \mathscr{F} \set{ \exp(3jt - |2t + 2|) } = \frac{4 e^{-j(\omega - 3)}}{(\omega - 3)^2 + 4}
} \]
\end{solution}
% ------------- %

\newpage

% --- 3.2.3 --- %
\begin{subsubproblems}[start=3]
    \item \( \frac{1}{2\pi t^2 + 1} \)
\end{subsubproblems}

\begin{solution}
Consider the CFTF of \( e^{-|t|} \) obtained in Problem 3.1:
\[
    \mathscr{F} \set{ e^{-|t|} } = \frac{2}{\omega^2 + 1}
\]

Using the Duality property of the Fourier transform, we have:
\[
    \textbf{Duality: } \mathscr{F} \set{ X(t) } = 2\pi x(-\omega)
\]

Define a new signal \( y(t) = \frac{2}{t^2 + 1} \). Then, applying the Duality property to \( y(t) \), we get:
\begin{align*}  
    Y(j\omega) &= \mathscr{F} \set{ y(t) } \\
    &= 2\pi x(-\omega) \\
    &= 2\pi \mathscr{F}^{-1} \set{ X(t) } \bigg|_{t = -\omega} \\
    &= 2\pi e^{-|- \omega|} \\
    Y(j\omega) &= 2\pi e^{-|\omega|}
\end{align*}

Using the Time-scaling property of the Fourier transform, we have:
\[
    \textbf{Time-scaling: } \mathscr{F} \set{ x(at) } = \frac{1}{|a|} X\paren{ \frac{j\omega}{a} }
\]

We can rewrite the given signal as:
\[ \frac{1}{2\pi t^2 + 1} = \frac{1}{2} \frac{2}{ \paren{ \sqrt{2\pi} t }^2 + 1} = \frac{1}{2} y\paren{ \sqrt{2\pi} t } \]

Find \( y\paren{ \sqrt{2\pi} t } \) by substituting \( a = \sqrt{2\pi} \) into the Time-scaling property, we get:
\begin{align*}
    \mathscr{F} \set{ y\paren{ \sqrt{2\pi} t } } &= \frac{1}{|\sqrt{2\pi}|} Y\paren{ \frac{j\omega}{\sqrt{2\pi}} } \\
    &= \frac{1}{\sqrt{2\pi}} \cdot 2\pi e^{- \abs{ \frac{\omega}{\sqrt{2\pi}} }} \\
    \mathscr{F} \set{ y\paren{ \sqrt{2\pi} t } } &= \sqrt{2\pi} e^{- \frac{|\omega|}{\sqrt{2\pi}} }
\end{align*}

Therefore, we can express the Fourier transform of \( \frac{1}{2\pi t^2 + 1} \) as:
\[ \boxed{
    \mathscr{F} \set{ \frac{1}{2\pi t^2 + 1} } = \frac{\sqrt{2\pi}}{2} e^{- \frac{|\omega|}{\sqrt{2\pi}} }
} \]
\end{solution}
% ------------- %
% =========== %
% ================================================================================ %

\end{document}